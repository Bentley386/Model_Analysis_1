\documentclass{article}

\usepackage[utf8]{inputenc}
\usepackage{amsmath}
\usepackage{graphicx}
\usepackage{subcaption}
\usepackage{caption}
\usepackage{float}
\usepackage{geometry}
\usepackage{physics}
\geometry{margin=1in}

\title{Nelinearna minimizacija}
\author{Andrej Kolar-Požun, 28172042}



\errorcontextlines 10000
\begin{document}
\pagenumbering{gobble}
\maketitle
\newpage
\pagenumbering{arabic}

\section{Thomsonov problem}
Zanima nas porazdelitev N elektronov po prevodni sferni lupini. To bomo dobili z minimizacijo elektrostatske energije. To, da smo omejeni na sferno lupino enostavno upoštevamo tako, da gremo v sferne koordinate in fiksiramo $r=1$
\begin{align*}
&E(\vec{r}_1,\vec{r}_2,\vec{r}_3,...,\vec{r}_N) = \sum_{i<j} \frac{1}{|\vec{r}_i - \vec{r}_j|} \\
&E(\phi_1,\theta_1,...,\phi_N,\theta_N) = \frac{1}{\sqrt{2}}\sum_{i<j} \frac{1}{\sqrt{1-sin(\theta_i)sin(\theta_j)cos(\phi_i-\phi_j) - cos(\theta_i)cos(\theta_j)}}
\end{align*}

Sledi prikaz razporeditve nabojev po krogelni lupini. V nadaljnem bo N označeval število vseh nabojev, n število nabojev z nabojem 2(modra barva), m pa število nabojev z nabojem 10(rdeča barva). Nabojev z nabojem 1 je torej N-n-m(črna barva). 
\begin{figure}[H]
\centering
\begin{subfigure}{.49\linewidth}
\includegraphics[width=\linewidth]{vecN/1.pdf}
\end{subfigure}
\begin{subfigure}{.49\linewidth}
\includegraphics[width=\linewidth]{1.pdf}
\end{subfigure}
\end{figure}
\begin{figure}[H]
\centering
\begin{subfigure}{.49\textwidth}
\includegraphics[width=\linewidth]{2.pdf}
\end{subfigure}
\begin{subfigure}{.49\textwidth}
\includegraphics[width=\linewidth]{vecN/4.pdf}
\end{subfigure}
\end{figure}
\begin{figure}[H]
\centering
\begin{subfigure}{.49\textwidth}
\includegraphics[width=\linewidth]{vecN/5.pdf}
\end{subfigure}
\begin{subfigure}{.49\textwidth}
\includegraphics[width=\linewidth]{vecN/6.pdf}
\end{subfigure}
\end{figure}
\begin{figure}[H]
\centering
\begin{subfigure}{.49\textwidth}
\includegraphics[width=\linewidth]{vecN/7.pdf}
\end{subfigure}
\begin{subfigure}{.49\textwidth}
\includegraphics[width=\linewidth]{vecN/8.pdf}
\end{subfigure}
\end{figure}
\begin{figure}[H]
\centering
\begin{subfigure}{.49\textwidth}
\includegraphics[width=\linewidth]{vecN/9.pdf}
\end{subfigure}
\begin{subfigure}{.49\textwidth}
\includegraphics[width=\linewidth]{vecN/10.pdf}
\end{subfigure}
\end{figure}
\begin{figure}[H]
\centering
\begin{subfigure}{.49\textwidth}
\includegraphics[width=\linewidth]{vecN/11.pdf}
\end{subfigure}
\begin{subfigure}{.49\textwidth}
\includegraphics[width=\linewidth]{vecN/12.pdf}
\end{subfigure}
\end{figure}
\begin{figure}[H]
\centering
\begin{subfigure}{.49\textwidth}
\includegraphics[width=\linewidth]{vecN/13.pdf}
\end{subfigure}
\begin{subfigure}{.49\textwidth}
\includegraphics[width=\linewidth]{vecN/14.pdf}
\end{subfigure}
\end{figure}

\begin{figure}[H]
\centering
\begin{subfigure}{.49\textwidth}
\includegraphics[width=\linewidth]{vecN2/1.pdf}
\end{subfigure}
\begin{subfigure}{.49\textwidth}
\includegraphics[width=\linewidth]{vecN2/2.pdf}
\end{subfigure}
\end{figure}
\begin{figure}[H]
\centering
\begin{subfigure}{.49\textwidth}
\includegraphics[width=\linewidth]{vecN2/3.pdf}
\end{subfigure}
\begin{subfigure}{.49\textwidth}
\includegraphics[width=\linewidth]{vecN2/4.pdf}
\end{subfigure}
\end{figure}
\begin{figure}[H]
\centering
\begin{subfigure}{.49\textwidth}
\includegraphics[width=\linewidth]{vecN2/5.pdf}
\end{subfigure}
\begin{subfigure}{.49\textwidth}
\includegraphics[width=\linewidth]{vecN2/6.pdf}
\end{subfigure}
\end{figure}
\begin{figure}[H]
\centering
\begin{subfigure}{.49\textwidth}
\includegraphics[width=\linewidth]{vecN2/7.pdf}
\end{subfigure}
\begin{subfigure}{.49\textwidth}
\includegraphics[width=\linewidth]{vecN2/8.pdf}
\end{subfigure}
\end{figure}
\begin{figure}[H]
\centering
\begin{subfigure}{.49\textwidth}
\includegraphics[width=\linewidth]{vecN2/9.pdf}
\end{subfigure}
\begin{subfigure}{.49\textwidth}
\includegraphics[width=\linewidth]{vecN2/10.pdf}
\end{subfigure}
\end{figure}
\begin{figure}[H]
\centering
\begin{subfigure}{.49\textwidth}
\includegraphics[width=\linewidth]{vecN2/11.pdf}
\end{subfigure}
\begin{subfigure}{.49\textwidth}
\includegraphics[width=\linewidth]{vecN2/12.pdf}
\end{subfigure}
\end{figure}
\begin{figure}[H]
\centering
\begin{subfigure}{.49\textwidth}
\includegraphics[width=\linewidth]{vecN2/13.pdf}
\end{subfigure}
\begin{subfigure}{.49\textwidth}
\includegraphics[width=\linewidth]{vecN3/1.pdf}
\end{subfigure}
\end{figure}
\begin{figure}[H]
\centering
\begin{subfigure}{.49\textwidth}
\includegraphics[width=\linewidth]{vecN3/2.pdf}
\end{subfigure}
\begin{subfigure}{.49\textwidth}
\includegraphics[width=\linewidth]{vecN3/3.pdf}
\end{subfigure}
\end{figure}
\begin{figure}[H]
\centering
\begin{subfigure}{.49\textwidth}
\includegraphics[width=\linewidth]{vecN3/4.pdf}
\end{subfigure}
\begin{subfigure}{.49\textwidth}
\includegraphics[width=\linewidth]{vecN3/5.pdf}
\end{subfigure}
\end{figure}
\begin{figure}[H]
\centering
\begin{subfigure}{.49\textwidth}
\includegraphics[width=\linewidth]{vecN3/6.pdf}
\end{subfigure}
\begin{subfigure}{.49\textwidth}
\includegraphics[width=\linewidth]{vecN3/7.pdf}
\end{subfigure}
\end{figure}
\begin{figure}[H]
\centering
\begin{subfigure}{.49\textwidth}
\includegraphics[width=\linewidth]{vecN3/8.pdf}
\end{subfigure}
\begin{subfigure}{.49\textwidth}
\includegraphics[width=\linewidth]{vecN3/9.pdf}
\end{subfigure}
\end{figure}
\begin{figure}[H]
\centering
\begin{subfigure}{.49\textwidth}
\includegraphics[width=\linewidth]{vecN3/10.pdf}
\end{subfigure}
\begin{subfigure}{.49\textwidth}
\includegraphics[width=\linewidth]{vecN3/11.pdf}
\end{subfigure}
\end{figure}
\begin{figure}[H]
\centering
\begin{subfigure}{.49\textwidth}
\includegraphics[width=\linewidth]{vecN3/12.pdf}
\end{subfigure}
\begin{subfigure}{.49\textwidth}
\includegraphics[width=\linewidth]{vecN3/13.pdf}
\end{subfigure}
\end{figure}
\begin{figure}[H]
\centering
\begin{subfigure}{.45\textwidth}
\includegraphics[width=\linewidth]{vecN3/14.pdf}
\end{subfigure}
\begin{subfigure}{.45\textwidth}
\includegraphics[width=\linewidth]{vecN3/15.pdf}
\end{subfigure}
\begin{subfigure}{.45\textwidth}
\includegraphics[width=\linewidth]{vecN3/16.pdf}
\end{subfigure}
\end{figure}

Rezultati so pričakovani, naboji so istega predzanaka in želijo biti za minimalno energijo čim dlje drug od drugega. Če imamo prisoten večji naboj, ta še bolj odbija ostale.
\newpage
Naslednja grafa prikazuje odvisnost energije in hitrosti izvajanja za več števil elektronov N. Primerjam Powellovo in Nelder-Mead(amebo) metodo.

\begin{figure}[H]
\centering
\begin{subfigure}{\textwidth}
\includegraphics[width=\linewidth]{tom2.pdf}
\end{subfigure}
\caption*{Nelder-Mead dela veliko počasneje(desna skala), zahtevnost pa pri obeh hitro narašča z N}
\end{figure}

\begin{figure}[H]
\centering
\begin{subfigure}{\textwidth}
\includegraphics[width=\linewidth]{tom3.pdf}
\end{subfigure}
\caption*{Poleg počasnosti je Nelder-Mead ponekod tudi občutno manj natančna, medtem ko se Powell zelo približa pravim vrednostim(Iz Wikipedije)}
\end{figure}


Kako pa se elektroni razporedijo po eliptični površini?
Problema se bom lotil v eliptičnih koordinatah(izberem $a=\mu=1$):
\begin{align*}
&x = a*cosh(\mu) cos(\theta) cos(\phi) \\
&y = a*cosh(\mu)cos(\theta)sin(\phi) \\
&z = a*sinh(\mu)sin(\theta) \\
&E = \sum_{i<j} (ch^2\mu(cos^2\theta_i+cos^2\theta_j - 2cos\theta_icos\theta_jcos(\phi_i - \phi_j)) + sh^2\mu(sin\theta_i - sin\theta_j)^2)^{-1/2}
\end{align*}

\begin{figure}[H]
\centering
\begin{subfigure}{.49\textwidth}
\includegraphics[width=\linewidth]{elipsoidi/elipsa1.pdf}
\end{subfigure}
\begin{subfigure}{.49\textwidth}
\includegraphics[width=\linewidth]{elipsoidi/elipsa2.pdf}
\end{subfigure}
\end{figure}
\begin{figure}[H]
\centering
\begin{subfigure}{.49\textwidth}
\includegraphics[width=\linewidth]{elipsoidi/elipsa3.pdf}
\end{subfigure}
\begin{subfigure}{.49\textwidth}
\includegraphics[width=\linewidth]{elipsoidi/elipsa4.pdf}
\end{subfigure}
\end{figure}
\begin{figure}[H]
\centering
\begin{subfigure}{.49\textwidth}
\includegraphics[width=\linewidth]{elipsoidi/elipsa5.pdf}
\end{subfigure}
\begin{subfigure}{.49\textwidth}
\includegraphics[width=\linewidth]{elipsoidi/elipsa6.pdf}
\end{subfigure}
\end{figure}

\begin{figure}[H]
\centering
\begin{subfigure}{\textwidth}
\includegraphics[width=\linewidth]{tom4.pdf}
\end{subfigure}
\caption*{Elipsoid ima na voljo več površine in se lahko elektroni razporedijo bolj narazen.}
\end{figure}

\section{Semafor}

Hitrosti $v_i$ bom iskal na diskretiziranem intervalu časa $t = i\Delta t ; i=0,1..,N$ z robnimi pogoji $v_0=1, v_{N} = 3$.
Lagrangian $L = \int \dot{v}^2 dt$ bom zapisal s pomočjo trapezne formule za numerično integracijo in to poskušal minimizirati.
Moji vezi, $\int v dt = L$ in $v < v_{max}$ bom upošteval tako, da bom moji funkciji prištel se ustrezne člene, parameter lambda pa bom iskal numerično s funkcijo fsolve:
\begin{align*}
&F_1 = 0.5\Big(\frac{v_1 -v_0}{\Delta t}\Big)^2 + \Big(\frac{v_2- v_0}{\Delta t}\Big)^2 + ... + 0.5\Big(\frac{v_N-v_{N-1}}{\Delta t}\Big)^2\\
&F_2 =  e^{\alpha (v_1 - v_{max})} +  e^{\alpha (v_2 - v_{max})} + ... +  e^{\alpha (v_{N-1} - v_{max})} \\
&F_3 = -\lambda (v_0/2 + v_1 + v_2 + ... + v_{N-1}/2)\Delta t \\
&F_1 + F_2 + F_3 = min
\end{align*}

Za naslednje grafe bom postavil $t_0=v_0=1, v_N = 3$, kar je nekako ekvivalentno spreminjaju c-ja v prvi nalogi. Spreminjal bom razdaljo do semaforja L. Za N pa sem vzel 50(Če je premajhen se pojavijo težave pri iskanju optimalnega lambda).

\begin{figure}[H]
\centering
\begin{subfigure}{\textwidth}
\includegraphics[width=\linewidth]{semafor2.pdf}
\end{subfigure}
\caption*{Tukaj so pogoji takšni, da je že tako ali tako optimalna vožnja pod maksimalno hitrostjo, pa jo alfa vseeno malo zniža}
\end{figure}



\begin{figure}[H]
\centering
\begin{subfigure}{\textwidth}
\includegraphics[width=\linewidth]{semafor.pdf}
\end{subfigure}
\caption*{V tem primeru so parametri taki, da bi za minimizacijo samo funkcionala bilo ugodno voziti tudi čez maskimalno hitrost. Za znižanje hitrosti pod maskimalno $\alpha=1$ ni dovolj.}
\end{figure}

\begin{figure}[H]
\centering
\begin{subfigure}{\textwidth}
\includegraphics[width=\linewidth]{semafor3.pdf}
\end{subfigure}
\caption*{Sedaj pa moramo, če želimo voziti pod maksimalno hitrostjo takoj močno pospešiti in na koncu zavirati.}
\end{figure}

\end{document}

